\documentclass[11pt]{article}
\usepackage{fullpage}
\usepackage{epsfig}
\usepackage{algorithm}
\usepackage[noend]{algorithmic}
\usepackage{amsmath,amssymb,amsthm}
\usepackage{enumerate}
\usepackage{framed}
\usepackage{titlesec}
\usepackage{listings}
\usepackage{tabto}
\usepackage{xcolor}
\usepackage{listings}
%\usepackage{lstlisting}
\usepackage{hyperref}
\usepackage{multicol}

\usepackage{graphicx}
\usepackage{caption}
\usepackage{subcaption}

\usepackage{enumitem}

% FILL IN THE SPECIFICS OF EACH HOMEWORK HERE
\newcommand{\course}{Research Project}
\newcommand{\semester}{Fall 2020}
\newcommand{\name}{Researh Project}
%%%
%%%
%%% PLEASE FILL OUT YOUR NAME AND THE HOMEWORK/PROBLEM NUMBER
%%% DO NOT INCLUDE STUDENT NAMES, FOR ANONYMOUS PEER GRADING PURPOSES.
%%%
%%%
\newcommand{\hwk}{Learning Notes}

\newcommand{\reportname}{Gem5 Learning Notes}


\usepackage[margin=3cm]{caption}


%%% You can ignore the following stuff, it's just for formatting purposes
\textheight=8.6in
\setlength{\textwidth}{6.44in}
\addtolength{\headheight}{\baselineskip}
% enumerate uses a), b), c), ...
\renewcommand{\labelenumi}{\alph{enumi})}
% Sets the style for fancy pages (namely, all but first page)
% Changes the title box on the first page
\renewcommand\maketitle{
\begin{center}
  {\huge\bf\reportname}\\
\begin{tabular*}{6.44in}{l @{\extracolsep{\fill}}c r}
%\bfseries  &  & \bfseries \course ~\semester \\
%\bfseries&  & \bfseries  \hwk  \\
\bfseries   &   &  \bfseries \\
\bfseries   &   &  \bfseries \\
Name: Haoxuan Zhang \bfseries  &  & \bfseries \course ~\semester\\
 \bfseries&  & \bfseries  \hwk
\end{tabular*}
\end{center} }


%% adjustments for dynamic programming.

%\titleformat{\subsection}{\bf}{Part \thesubsection:}{1ex}{}
\titleformat{\section}[runin]{\normalsize\bfseries}{\thesection.}{5pt}{}
%\renewcommand{\thesubsection}{\alph{subsection}}
\titleformat{\subsection}[runin]{\normalsize\bfseries}{\-\hspace{0.5cm}\thesubsection}{5pt}{}
%\titleformat{\section}{\wrap}{}{}{}{}



%%
%%
%% THE REAL STUFF STARTS HERE
%%
%%
\begin{document}
\maketitle
\titlelabel{\thetitle.\quad}

\begin{multicols}{2}[\section{Learning Notes for O3 CPU}]

\subsection{O3 CPU Attributes}
	\begin{enumerate}
		\item From Base CPU:\\
		
		protected:
			\begin{enumerate}[label={\arabic*.}]
				\item instCnt
				\item \_cpuId
        \item socketId
        \item taskID
        \item \_pid
        \item bool \_switchedOut
        \item \_cacheLineSize
        \item \texttt{std::vector\textless BaseInterrupts*\textgreater interrupts;}
        \item \texttt{std::vector\textless ThreadContext *\textgreater threadContexts;}
        \item \texttt{Trace::InstTracer * tracer;}
        \item \texttt{Cycles previousCycle;}
    		\item \texttt{CPUState previousState;}
			\end{enumerate}
		public:
			\begin{enumerate}[label={\arabic*.}]
				\item \texttt{ThreadID numThreads;}
				\item \texttt{System *system;}
			\end{enumerate}
			
		\item From FullO3CPU:\\
		
		protected:
			\begin{enumerate}[label={\arabic*.}]
				\item EventFunctionWrapper tickEvent; The tick event used for scheduling CPU ticks. What is a CPU tick?
				\item EventFunctionWrapper threadExitEvent;The exit event used for terminating all ready-to-exit threads
				\item \textbf{typename CPUPolicy::Fetch fetch;} This is a class which needs Impl specified
				\item \textbf{typename CPUPolicy::Decode decode;} This is a class which needs Impl specified
				\item \textbf{typename CPUPolicy::IEW iew;} The issue/execute/writeback stages.
				\item \textbf{typename CPUPolicy::Commit commit;} The commit stage.
				\item \textbf{PhysRegFile regFile;} The physical Register File
				\item \textbf{typename CPUPolicy::FreeList freeList;} What is this?
				\item \textbf{typename CPUPolicy::RenameMap renameMap[Impl::MaxThreads];} The rename map
				\item \textbf{typename CPUPolicy::RenameMap commitRenameMap[Impl::MaxThreads];} The commit rename map
				\item \textbf{typename CPUPolicy::ROB rob;} The re-order buffer.
				\item \textbf{std::list\textless ThreadID\textgreater activeThreads;} Active Threads List
				\item \textbf{std::unordered\_map\textless ThreadID, bool\textgreater exitingThreads;} This is a list of threads that are trying to exit. Each thread id is mapped to a boolean value denoting whether the thread is ready to exit.
				\item \textbf{Scoreboard scoreboard;} The scoreboard
				\item \textbf{TimeBuffer\textless TimeStruct\textgreater timeBuffer;} The main time buffer to do backwards communication.
				\item \textbf{TimeBuffer\textless FetchStruct\textgreater fetchQueue;} The fetch stage's instruction queue.
				\item \textbf{TimeBuffer\textless DecodeStruct\textgreater decodeQueue;} The decode stage's instruction queue.
				\item \textbf{TimeBuffer\textless RenameStruct\textgreater renameQueue;} The rename stage's instruction queue.
				\item \textbf{TimeBuffer\textless IEWStruct\textgreater iewQueue;} The IEW stage's instruction queue.
				\item \textbf{ActivityRecorder activityRec;} The activity recorder; used to tell if the CPU has any activity remaining or if it can go to idle and deschedule itself.
				

			\end{enumerate}
			
		private:
		\begin{enumerate}[label={\arabic*.}]
			\item \textbf{System *system;} Pointer to the system.
			\item \textbf{std::map\\\textless ThreadID,unsigned\textgreater threadMap;} Mapping for system thread id to cpu id
			\item \textbf{std::vector\textless ThreadID\textgreater tids;} Available thread ids in the cpu
			\item lots of Stats at the end of the class declaration
		\end{enumerate}
			
			
		public:
			\begin{enumerate}[label={\arabic*.}]
				\item \texttt{    enum Status \{\\
        						\-\hspace{1cm}	Running,\\
        						\-\hspace{1cm}	Idle,\\
        						\-\hspace{1cm}	Halted,\\
        						\-\hspace{1cm}	Blocked,\\
        						\-\hspace{1cm}	SwitchedOut\\
    												\};}
    		\item \texttt{BaseTLB *itb;}
    		\item \texttt{BaseTLB *dtb;}
    		\item \texttt{Status \_status;}
    		\item \texttt{int instcount;} with flag \texttt{NDEBUG} Count of total number of dynamic instructions in flight.
    		\item \texttt{std::list<DynInstPtr> instList;} List of all the instructions in flight.
    		\item \texttt{std::queue<ListIt> removeList;} List of all the instructions that will be removed at the end of this cycle.
    		\item \texttt{bool removeInstsThisCycle;} Records if instructions need to be removed this cycle due to being retired or squashed.
			\end{enumerate}
	\end{enumerate}
\subsection{Sept 19 2020}
\begin{enumerate}
  \item What is LSQ request in Full O3 CPU?
  \item if there are multiple lanes dispatching at the same time. how should they register themselves in the reorder buffer?
  \item DefaultIEW is initializing width in the way: issueWidth(params-\textgreater issueWidth) Maybe changing such parameters could change the configuration. next problem is how to change the bandwidth of the memory
  \item It seems that the SimObject is polymorphic so I could convert a point to SimObject to a pointer to the derived class.
  \item seems that all the parameters in the python classes are declared not in \_\_init\_\_() but in the class itself.
  \item what is a probe?
  \item every sim\_object has a name and the plan is to split the name and check whether the name ``cpu'' or something is in the list of strings. If so, the object is found.
\end{enumerate}
\end{multicols}
\end{document}
